\documentclass[12pt,oneside,fleqn]{book}
\usepackage[margin=1in]{geometry}
\usepackage{amsfonts,amsmath,amsthm,amssymb}
\usepackage{graphicx,hyperref,xstring,enumitem}
\usepackage{qtree}

\usepackage{xcolor}
\usepackage{listings}
\lstset{basicstyle=\ttfamily,
  showstringspaces=false,
  commentstyle=\color{red},
  keywordstyle=\color{blue}
}

\setlength{\mathindent}{0pt}
\setlength{\textheight}{9.5in}
\setlength{\textwidth}{7.0in}
\setlength{\topmargin}{-0.75in}
\setlength{\oddsidemargin}{-0.25in}
\setlength{\evensidemargin}{0.75in}
\setlength{\parskip}{0.15in}
\setlength{\parindent}{0in}

\newcommand{\brac}[1]{\left( #1 \right)}

\begin{document}
\thispagestyle{empty}
\subsection*{CS 246 NOTES}
Notes typeset and compiled by Joey Pereira \\
\vspace{1in}
\\
\underline{Disclaimer:}
These notes are not meant to be a replacement for lecture. They are meant to be a method of learning for myself for \LaTeX, and are meant to be used for reference or review of content. I share these in hopes that it helps others understand or review the content covered.
\newpage

\noindent
\tableofcontents
\thispagestyle{empty}
\clearpage
\newpage

\pagenumbering{arabic} 
\setcounter{page}{1}
\chapter{Introduction to Linux}
\section{History of the Shell}
\begin{itemize}
\item Unix Shell (made in the 70's) called "shell" written by Bourne
\item Other random shells made at the time, csh (c shell), ksh (korn shell)
\item csh was revised into turbo shell3
\item sh was named Bourne shell after many shells began appearing
\item Bourne Again shell (bash) was created as a replacement for Bourne Shell
\end{itemize}



\section{Linux File System}
Composed of
\begin{itemize}
\item files (programs, data)
\item directories (files/directories)
\end{itemize}
Directories form a tree-like structure
\Tree[./ [.usr [.bin ] [.lib ]]
         [.home [.joey [.documents ] ] ]
         [.root ]]

\newpage
The initial or \emph{root directory} of the tree or filesystem is known as ``/"
Absolute paths to a file in the file system always start with the root directory \\
Example: ``/home/joey/document.txt", ``/usr/bin/firefox"

The \emph{Current directory} is directory you are currently sitting in, which you can see by running
\begin{lstlisting}[frame=single,language=bash]
	$ pwd
	\end{lstlisting}
which displays the absolute directory currently.
\emph{Relative paths} don't start with a ``/".
Examples: ``joey/documents", ``./firefox" \\

\subsection{Special Directories}
\begin{itemize}
\item "." $\to$ current directory
\item ".." $\to$ refers to parent of current directory
\item "{\raise.17ex\hbox{$\scriptstyle\mathtt{\sim}$}}" $\to$ your home directory, also you can just do "cd" to get to home
\item "{\raise.17ex\hbox{$\scriptstyle\mathtt{\sim}$}}userid" $\to$ is that users home directory
\end{itemize}

\section{Linux Programs}
Where ever you are while in directories, you can run programs via the command line.
\
\begin{itemize}
\item does not show any file starting with a dot
\item using "ls -a" will list out all of the files including hidden ones
\item Wildcard matching: \\
{\it aside:} in linux there is no enforcement of what the extension means for a file e.g. binaries could be .txt, no special meaning \\
{\it example:} wanting to list files that end in .txt in the current directory
"ls *.txt", "*.txt" is called a globbing pattern \\
Shell substitues *.txt with every file in the current directory that matches the pattern onto the command line
\end{itemize}

echo $\implies$ simply echos whatever you type back \\
rm $\implies$ removes (deletes) files \\
cat $\implies$ concatenates the files listed \\

Ctrl + C to kill a process 

note that when we do cat it just repeats what we type \\
It would be useful if we can capture it \\
done by cat $>$ nameofafile.txt \\
But we need to do a clean exit, so Ctrl + D does an EOF to thevim program

Output redirection \\
Using $<$ and also $>$ and whatnot \\
Every program has 3 streams \\
Standard input, standard output and standard error \\
The standard output and error defaults to stream \\
Default input defaults to linked to keyboard \\
To redirect error e.g. program.exe 2$>$ error.log \\

Permissions:
3 groups of 3 bits
example: rwx r-x r--
first 3 are what the owner can do
second iwhat members of the group (except owner) can do
other bits what all can do
Inside each group we have a read bit, a write bit, and an execute bit. If the bit/permission is not set, it appears as a dash.

Permissions:
for files
read - read the content of the file
write - modify the content of the file
execute - execute the file as a program

for folders
read - list the contents of the folder, globbing, tab completion
write - add or remove files in the directory
execute - directory can be navigated to e.g. cd

Changing permissions
only the owner can change permissions
chmod  mode file
mode ownership-class operators					permissions
     u-user			  + add						r
     g-group		  - remove					w
     o-other		  = set permission exactly	x
     a-all


example: give others the ability to read permissions
chmod o+r file
revoke execute permissions from group
chmod g-x file
make everyones permissions rx
chmod a=rx file

scripts are text files containing a sequence of linux commands execured as a program
example: print the date, current user, current directory
\# !/bin/bash
date
whoami
pwd

if current directory i not in \$PATH, run script as ./script

redirectoing to /dev/null will delete any buffer (black whole)

programs return a status code
0 - successfull
non-zero - failure
\$? is set to the status code
\$0 is the name of the currently running file

Looping over a list
ex.
REname all .cpp files to .cc
\#\!/bin/bash
for name in *.cpp; do
	mv \${name} \${name\%cpp}cc
done

ex.
How many times does \$1 occur in \$2 (file)
\# !/bin/bash
x=0
for word in `cat \$2`; do
	if [ \$word = \$1 ]; then
		\$x=\$(x+1)
	fi
done
echo \$x

Good idea to put input variables in double quotes to code defensively so space delimited input is never executed

Testing
- no one likes testing
- create test suite before actual code
Functional Testing
- run every function at least once
- within a function
	- cover different control flow
	- code coverage
	- variable ranges: 	positive/negative
						boundaries of ranges (edge cases)
						multiple simultaneous boundaries (corner case)

Regression Testing
- new code does not break old code


\section{C++}
Bjorne Stroustrup
-invented the idea of a class and created "C with classes"

How can we tell an attempted read failed?
If a read fails then cin.fail() is true
if a read fails because EOF then cin.fail() is true and cin.eof() is true.

Example:
Read all integers from stdin and echo to stdout
Stop if non-integer or EOF is encountered.
There is an implicit conversion from cin (iostream) to void *
Since a pointer is some numeric value, it can be used in a condition
Thus we can use just cin inside a condition
cin is true if !cin.fail()

We are using >> as the "get from" (input) operator 
Which also does bit shifts.
Also cin >> a is an expression that returns cin
When a read fails, a flag goes up in cin
- the flag stays raised until we acknowledge a failure
cin.clear() - lowers the flag
cin.ignore() - skips over to the next character

% queue le visha leave
C++ provides the type std::string for helping with string stuff (such as reading in strings)

% queue le readStrings.cc
this will read in characters until it hits a whitespace, then its done.
i.e. go past any whitespaces until it hits the first whitespace
Reads the next characters into a String, stopping at the next whitespace, ignoring any leading whitespace.
Aside: to read the entire line (including whitespace) getline(cin, s)
cin for strings is similar to scanf with \%s format specifiers in C

Notice that 
cin >> i reads in an integer
cin >> s reads in a string
The exact same code to read either in, the only difference is the declared type (e.g. type of i or s variable)

C-style format specifiers do have their benefits
int i = 95;
cout << i << endl; // just prints out 95 in decimal
What if we want hexidecimal, binary, boolean (true/false literal)?
We use I/O manipulators to do that. Examples\\
cout << hex << i << endl; (note hex is std::hex, we are working with namespace std)
Sending hex to cout does not print anything. It makes cout change in  way that says print any integers that are sent in as hexidecimal form
Unless we do cout << dec, to change the behaviour back to decimals

Moving along, what if we don't want to read in from keyboard (stdin).
This "stream" abstraction we've been using as cin (specifically istream and ostream - in and out respectively) we can use the same way to open and "stream" a file.
To read or write from files, we have file streams. They behave just like input and output streams.

% le queue fileRead.cc
The only difference in this example is how we initialize the file stream, requiring a file name!
ifstream f("suite.tx"); //variable declaration and initalization
type: ifstream
varaible name: f
parameters for initializing: "suite.txt" (string of filename)

We can attach a stream to a string and read/write from it
include <sstream>
So we have an istringstream -read from a string
And ostringstream - write to a string

int low...
int high...
ostringstream ss;
ss << "Enter an integer between << lo << " and " << high;

Default Arguments:
in c++ we can specify default arguments
void printSuiteFile(string filename="suite.txt) <- is the default value

string s = ss.str();
cout << s << endl;

Benefit: Convert a number to a string
How about reading a string?
We can use an ostringstream

Strings in C++
In C we didn't have a built in string type, so instead we used arrays of characters.
- needed careful memory management
- was very easy to overwrite the null terminator

In C++ we do have a string type
- not built in, but part of standard library
std::string type
-manage their own memory
- safer to use
string s = "sdgd";
can assign c string to a c++ string and c++ does a conversion

equality: c++ has all of the comparison type operators (==, >= <=)

ability to refer to individual characters
In c++ we can still do s[0], s[1] for characters of a string

string concating
in C: strcat
in c++
s1 + s2 works!

Note: to get a c style string from a c++ string, we need to use string.c\_str()

Overloading:
Example:
write functions that negate integers and one that negates bools

Pointers
int n = 5;
int *p = \&n;
cout << p << endl; //prints the pointer to n
cout << *p << endl; // prints the int 5

int **pp;
pp is a pointer to a pointer to an int.
pp = \&p;

Arrays
Not to be confused with pointers
int a[] = { 1, 2, 3, 4}
this way has a few restrictions of whats needed. 
the array needs a contiguous block of memory
The name of the array is short form for the address of the first element in the array.
a = \&a[0]
*a = a[0] -> first element

Structs

C syntax
Struct Node {
	int data;
	Struct Node * next;
};

Constants
const int maxGrade = 100;
Make as many tihngs constance as possible
A constant must always be initialized

const int * p = \&n;
p is a pointer to a constant integer
what you cant do is change the value of n through p

int * const p = \&n p is a constant pointer to an integer

cin behaves similarly to scanf
for scanf we pass the address of the variable we want to read into
although cin >> x takes in not a pointer, it still acts as if it was given one

CPP has another pointer like type - a refernce
int y = 10;
int \&z = y;
z is a reference to y
z is a constant pointer to y

a reference is a constant pointer with automatic dereferencing
z = 12; (NOT *z = 12 - but this 'does' the same thing)

int *p = \&z
What happens here even is that p gets the address of y. In no case does z have an identity of it's own.

Cannot have a reference uninitialized
its a constant pointer -> constant -> must be initialized

Can only create a reference to something that has an address.
Note: anything that has an address is called an lvalue

Cannot create a reference to a pointer
Cannot create a reference to a reference
Cannot create a reference to an array
Cannot create a reference to a pointer


Pass by Value
struct ReallyBig
f(ReallyBigStruct)
results in copying over the entire struct
In C we could pass a pointer to suppress the entire copy
In CPP we can also use a reference to do similarly.

void h (const ReallyBig \&rb) will ensure rb does not get changed outside of scope.

Dynamic Memory Allocation
CPP provides new and delete for allocation and deallocating.
New is type aware, so it's less prone to errors

Node *np = new Node;
as simple as that
Since it is type aware it knows how much memory is needed for Node
Once done, to reclaim memory
delete np;

Dynamically allocating an array
cin >> n;
int * arr = new int[n], number of elements in the array

Once done, to deallocate the array
delete [] arr;

Operator overloading
Give a special meaning to C++ operators for types we construct



% thursday june 5th
\newpage
Preprocessor
- a program intercepts the code before it is compiled
- transforms code

\#include <iostream>
this is a preprocessor directive
This tells the preprocessor to paste the contents of iostream right here
Look for iostream in the standard place that contains the cpp library

Also
\#include ``..."
- look for this file in my current directory

Another directive
\#define VAR VALUE
tells the preprocesor to setup a variable called VAR with the value VALUE

The define directive orders the preprocessor to replace any occurrence of VAR with VALUE
-does a search and replace

\#define MAX 10
int array[MAX]; => int array[10];

Define is a mechanism to specify constants.
It's completely useless now as C and CPP now have const variable types which made define obsolete.

Preprocessor does nto know (or care) what would get generator by doing the "FIND AND REPLACE" of define.

Define constants are useful for conditional compilation
Examples:
In unix it might be
int main()
While windows might require code to be
int WinMain()

So instead you can use preprocessing to just change the code immediately, as so \\
\#define Unix 1 \\
\#define Windows 2 \\
\#define OS Unix \\
\#if OS == Unix \\
int main { \\
\#elif OS == Windows \\
int WinMain() { \\
\#endif \\


\#define VAR
-define a variable VAR
- it's value is the empty string

\#if def VAR \\
blagh \\
\#ifndef VAR \\
blagh \\
\#endif


Note
doing
\#if 0
sadasfaf
afasf
asfasf
\#endif
-super duper way of commenting out code


Moreso we can specify defines from command line.
e.g.
g++ program.cc -DOS=Unix -o program
Will \#define OS Unix for you through the command line

Mind you if you do 
\#define DEBUG -> gets the empty string value
g++ -DDEBUG *.cc -> value is given 1

Separate compilation
-split programs into composable modules
Interface: .h type definitions, fucntion headers...
Implementation: .cc, actual implementation

Ways to compile
g++ *.cc (glob all files), if all .cc files are part of the program and you dont compile - header files included inside .cc
- never include cc file

g++ argument "-c" just compile. creates an object file, binaries for code
ld (linker) is the matchmaker that links the files together


Hard to keep track of whats being included
have IFDEFS in our headers!


Classes
Big innovation of OOP
You can put functions inside a struct 
A class is a struct type that can have functions
For now we will only regard classes as a struct which contains functions

Functions defined inside a class are called "member functions" or methods

Say we have a struct Student with a function grade();
When we access the structs fields inside it's methods, we may use "final" "midterm" as variables,
Where a when we access the method from the struct we are specifically detailing that structs instances

The distinction between fucntions and methods is that a method contains a hidden (implicit) parameter called "this" which is a pointer to which the object that this pointer was called


Initializing Structs
Student billy = { 60, 70, 80}
C style initialization of an object
- restrictive

CPP provides th abiity to write constructors which are functions used to construct objects

Struct Studenet {
	int assigs, mid, final;

	note fcn name is same as class name
	Student (int assigs, int mid, int final) {
	this.assigs = assigs;
	this.mid = mid;
	this.final = final;
	}
}

Now using our constructor
Student billy(60,70,80); 
Student billy = Student(60,70,80);

And on the heap
Student * pBilly = new Student(60,70,80);

Advantages of writing constructors
- full CPP functionality is available
-default parameter values
Student(int assigs = 0, int mid=0, int final=0) {
	regular constructor
}


If you want to default all parameters (so pass nothing)
You want to do
Person per;
and it uses all of the default parameters for the constructor


% thurs jun 12
Initializing Objects
every class comes with a default no argument constructor.
- calls default constructors on fields (if they have constructors)

e.g.
Vec v; // default constructor ran

Note: the builtin (default) constructor goes away as soon as you write any constructor for the class

Once we've defined a struct with our own constructor, we can no longer do


Struct Vec {
	int x;
	int y;
	
	Vec (int xx, int yy = 2) {
		this->x = xx;
		this->y = yy;
	}
};

Vec v1(1,2);
Vec v2 = new Vec(1,2);
Vec v(1);




Vec v; // won't compile
Vec v (1,2); //ok!
Another thing you loose when you build your own constructor is c-style initializing
Vec v = { 1,2} ; // wont compile

Q: What if my class has constant as reference fields?
struct MyStruct{
	const int myConst;
	int \&myref;
};

COnstants and references must be initalized!
Ok, to initalize them


struct MyStruct{
	const int myConst = 5;
	int \&myRef = z;
};


inializers are not allowed though for fields! (won't compile)
What you most likely want is that each object gets it's own constant independant of the class
Each instance of myStruct gets its own myConst and myRef
Why should they have the same value?

Q: Can we initliaze these inside the construcotr body?
A: No, it's too late. by the time the constructor body executes constants and references should have already been initialized




Struct Person {

	int x;
	int y;
	Struct Wallet w;
	

};

new Person();
Wallet w = new Wallet();



When an object is created,
1) space is allocated (stack or heap)
2) fields are initialized: default constructors are executed for all fields that have default constructors
3) constructor body runs

hijack step 2:
done by a member initialization list

Struct MyStruct {
	const int myConst;
	int \&myref;
	
	int x;
	
	MyStruct(int c, int \&r, int xx): myConst(c), myref(r), x(xx) {
	~~~~ CONSTRUCTOR ~~~~
	}
};
-This sort of syntax is only allowed for constructors
-You are not restricted to just constants and references

Another example of the member list
Struct Student{
	int assigs, mid, final;
	Student(int assigs, int mid, int final): assigs(assigs), mid(mid), final(final) {
	}
};

In this case note that in field(param), FIELD is ALWAYS just looked at as a field, and PARAM is ONLY looked at as parameters in this syntax
Another advantage of member initialization lists is that they [can] be more efficient then initializing inside the constructor body (due to default initilization being run and initializing all fields twice if re initializing in cosntructor body)
[ if all the fields are primative types they won't be initialized and it won't make a difference anyways as primate initiliazation isn't much ]

Fields are initialized in the order in which they are declared
-irrespective of the order appear in the in the members initialization list
-most compilers give you a warning if your order in the initialization list was different than the order of declaration


more initialization stuff
Student billy(60,70,80);
and we have bobby is a copy of billy
Student bobby = billy;
To construct a new object as a copy of an existing object we use the copy constructor

-you get a default copy ctor for free (makes a copy field to field)
Every class comes with:
-default constructor
-default copy constructor
- copy assignment operator
- destructor

Struct Student{
	int assigs, mid, final;
	Student(int assigs, int mid, int final): assigs(assigs), mid(mid), final(final) {
	}
	
	Student(const Student \&other): assigs(other.assigs), mid(other.mid), final(other.final) {}
};

Student bob(1,2,3);
Student billy(bob);

Default copy constructor might not do what you expect it to do

Struct Node{
	int data;
	Node * next;
	
	Node (int data, Node *next): data(data), next(next){}
	
	Node (const Node \&other) : data(other.data), next(other.next) {}
};

Node *n = new Node(1, new Node (2, new Node(3, 0)));

Node m = *n; // runs copy
Node *p = new Node (*n);

I wanted three linked lists that are copies of eachother
Unfortunately we only got NEW first nodes, but the rest of the list is the same for each list (points to the same elements)
We get whats called a shallow copy


DEEP COPY
Struct Node{

	Node(const Node \&others): data(others.data), next(other.next ? new Node(other.next) : 0) {}
	
};
\end{document}